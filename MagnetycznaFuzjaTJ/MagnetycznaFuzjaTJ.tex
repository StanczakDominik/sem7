\documentclass{article}
\usepackage[utf8]{inputenc}
\usepackage{polski}
\usepackage[polish]{babel}
\begin{document}
\title{Magnetyczna synteza termojądrowa}
\author{Dominik Stańczak}
\date{2016}

\maketitle
\begin{enumerate}
    \item Kilka słów o fuzji jądrowej
    \begin{enumerate}
        \item Dlaczego?
        \item Dlaczego nie?
    \end{enumerate}
    \item `Teoretyczne minimum' fizyki plazmy w odniesieniu do MFTJ
    \begin{enumerate}
        \item Czym jest plazma?
        \item Elektrodynamika cząstek
        % TODO
        \item lorem
        \item ipsum
        \item dolor
    \end{enumerate}
    \item Podejścia
    \begin{enumerate}
        \item Z-pinch
        \item $\theta$-pinch
        \item Zwierciadło magnetyczne
        \item Tokamak
            \item ITER
            \item JET
        \item Stellarator
            \item Wendelstein 7-X
        \item Reversed Field Pinch
        \item Levitated Dipole Experiment
        \item FRC (Field-Reversed Configuration)
        \item Sferomak
        \item Dynomak
        \item Tokamak sferyczny
        \begin{itemize}
            \item MAST
            \item NSTX-U
            \item Globus-M
        \end{itemize}
    \end{enumerate}
    \item Bibliografia
\end{enumerate}
\end{document}
